%*
%* Seven Kingdoms: Ancient Adversaries
%*
%* Copyright 1997,1998 Enlight Software Ltd.
%* Copyright 2018 Timothy Rink
%*
%* This program is free software: you can redistribute it and/or modify
%* it under the terms of the GNU General Public License as published by
%* the Free Software Foundation, either version 2 of the License, or
%* (at your option) any later version.
%*
%* This program is distributed in the hope that it will be useful,
%* but WITHOUT ANY WARRANTY; without even the implied warranty of
%* MERCHANTABILITY or FITNESS FOR A PARTICULAR PURPOSE.  See the
%* GNU General Public License for more details.
%*
%* You should have received a copy of the GNU General Public License
%* along with this program.  If not, see <http://www.gnu.org/licenses/>.
%*
%*

\chapter[Creating an Empire]{{\Huge C}REATING AN {\Huge E}MPIRE}

\index{creating!empire}

\section{\textsf{Settling New Villages}}

\index{villages!settling new}

\begin{wrapfigure}{r}{0.1\textwidth}
    \vspace{-20pt}
    \begin{center}
        \includegraphics[width=0.1\textwidth]{Tsettle}
    \end{center}
    \vspace{-20pt}
\end{wrapfigure}

\textswab{\huge{B}}y far, the simplest way to increase the extent of your realm is to Recruit a large group of peasants from one or several of your Villages and to send them into a suitable area to Settle. As all Villages have a Population limit of 60, it makes good sense to do this when a Village approaches this level.

\begin{wrapfigure}{l}{0.1\textwidth}
    \vspace{-20pt}
    \begin{center}
        \includegraphics[width=0.1\textwidth]{Bflag} % Original size.
    \end{center}
    \vspace{-20pt}
\end{wrapfigure}

After the Peasants have been recruited and are standing outside of their Village, select them all and then \textbf{Click} the \textbf{Settle Tile} (above right). Your cursor will immediately be changed into the Flag and Arrow Cursor (Right).

By placing this new cursor in any location on the map to which the Settlers are able to walk, you are ordering them to pick up and move to this new spot, build their new hovels, and raise the glorious colors of your Empire.

The \textbf{Settle Tile} may also be used to force skilled units to enter an existing village and revert to the life of a common Peasant. In this case, the unit(s) will retain their Combat rating, but forget any other skills they possess.

\subsection{\textsf{Controlling New Villages}}

\index{villages!controlling new}

\textswab{\huge{U}}nfortunately, Peasants far removed from your presence tend to grow rebellious, getting ideas of self determination and other such non-sense. After they have built their new Village, you will have no control over them, apart from the ability to Recruit them. (If your new Village is settled within Linking distance of an enemy Fort, you won’t even be able to Recruit your Peasants)

% Hyphenation here. and well-manicured?

These situations necessitate the erection and staffing of a new Fort Linked to that Village. Once finished, the wretches will once again be under your well-manicured thumb.

\section{\textsf{Taking over Enemy Villages}}

\index{villages!taking over enemy}

\textswab{\huge{A}} most satisfying experience can be had by humiliating a foe in battle and then taking one of his Villages as your own.

\subsection{\textsf{What Is the Best Way to Take an Enemy Village?}}

You must first destroy any and all Forts Linked to the Village that belong to the Kingdom that rules the Village.

Then you must build your own Fort, Linked to the Village and staffed with a General, preferably, though not necessarily, of the same Nationality as the Villagers.

You may then wait for the Village’s resistance level to your Kingdom to decrease on its own, or you may help it along with a liberal application of Grants or of violence. If you do attack, the Villagers whose resistance level is 50 or more will come out to fight. Village resistance will not begin to drop until all of those who have come out to fight have been dispatched.

When over two-thirds of the Villagers’ resistance level has dropped below 30, the Village will consider surrendering.

If there are Forts from other Kingdoms Linked to the Village, the Villagers will consider surrendering to the Kingdom with the highest Reputation and/or whom are of the same nationality as them.

It is important to note that when you attack a Village and slaughter civilians (no matter how deserving of such treatment they may be) your Reputation will suffer.

\section{\textsf{Absorbing Independent Villages}}

\textswab{\huge{I}}ndependent Villages, those with no flag flying, are ripe for the picking. They may be acquired by force, by persuasion, or by guile.

\textbf{Clicking} on an Independent Village, you will see, next to the Population and Peasants statistics, the Resistance Level to your rule. Your object is to lower this Resistance Level to zero for each nationality in that Village.

\subsection{\textsf{By Force}}

\index{villages!absorbing independent!by force}

\textswab{\huge{F}}orce is by far the fastest and most straightforward way of taking an Independent Village. Simply send an appropriately sized and trained army and attack. Those Villagers who possess a resistance level of 30 or more will take down their grandfather’s rusty swords and come out to fight, but as they are nothing but untrained farmers, they will soon be overwhelmed and leave the Village undefended. Hit the Village a few more times, and it will fall. Your banner will be raised, and with the erection and staffing of a Fort, you will take complete control.

In the rare instance that your forces are defeated by these Peasants, they will return to their Village as heroes, and the Village’s resistance to you will increase still further.

If this seems almost too easy to be true, it is. This method of Empire expansion has some serious drawbacks. Apart from killing most of the people in the Village and thereby rendering them unfit for useful labor, attacking peaceful Villages does your Reputation no end of harm. You may not care what others think of you, but a low Reputation rating can have disastrous implications.

And as your empire grows, you will come to have many nationalities under your control. When you attack an independent Village, it is quite likely that the targeted Villagers will be of a nationality shared by many of your subjects. This will certainly decrease their loyalty to you and could even ignite in them the spark of revolt.

\textbf{NOTE}: You may set the level of resistance possessed by these Villagers under the Advanced Options II section of the Game Settings Menu.

Independent Villagers may sometimes leave their Villages and join your Kingdom. If the Resistance Level is set to high in the initial game options, these Villagers will possess much greater Combat Skill than if you had set the Independent Village Resistance to low.

\subsection{\textsf{By Persuasion}}

\index{villages!absorbing independent!by persuation}

\subsubsection{\textsf{Building a Fort}}

\textswab{\huge{T}}he Building of a Fort Linked to an independent Village can be an effective way of imposing your will on these future subjects. It is best accomplished by appointing to the Fort a General of the same nationality as the Villagers. This will give them the illusion that you are so called sensitive to their situation and that you wish them only the best. Of course, as many Villages are made up of more than one nationality, a second Fort and another General may be advisable.

The rate at which the Resistance Level of a Village falls depends greatly upon the leadership ability of the General and on the Reputation of your Kingdom. The offering of Grants to the Villagers will also help to drive down their resistance.

You will be able to see the Resistance Level falling: a red, downward pointing arrow will appear between the present level on the left and the level towards which it is headed on the right.

The level on the right is the level to which the resistance will fall because of the effect of the Fort and the Leadership Level of its General. The level may fall still further, but that will depend on such things as your Reputation, the effect of Spies, and of your economic links (see below) to the Village.

If the resistance level falls to zero, the Village will gratefully become part of your empire. You may then use them as you will.

\subsubsection{\textsf{Providing Jobs}}

\textswab{\huge{T}}he Building of firms such as Mines, Factories, or anything that will give employment to local peasants, will help the Villagers to see the advantages of your way of life and subsequently help to lower resistance to your inevitable rule over that Village. It may not be enough by itself though, so this method is best used along with the building of a Fort.

\textbf{NOTE}: Villagers will not go to work in your Firms or buy in your Markets unless the average resistance level to your Kingdom is below 50. 

\subsubsection{\textsf{Results of Persuasion}}

\textswab{\huge{I}}ndependent Villagers and other units who have betrayed their Kings will occasionally migrate from their homes to more attractive places. It is up to you to make sure that your Kingdom is such a place. If you do this skillfully, you may acquire a huge number of people which can immeasurably strengthen your Empire.

You will be notified with a news message when these people have chosen to join you. They will stand outside of their old Villages and wait for your further orders. It is possible that some of these immigrants are in fact Spies in the pay of an enemy Kingdom. See \textbf{Chapter 17} for details on Spying.

On occasion, entire Independent Villages or enemy-controlled Villages may decide to join your Empire. They will only do this if your Kingdom has a very high Reputation and if they believe you are powerful enough to protect them.

\subsection{\textsf{By Guile}}

See next chapter.