%*
%* Seven Kingdoms: Ancient Adversaries
%*
%* Copyright 1997,1998 Enlight Software Ltd.
%* Copyright 2018 Timothy Rink
%*
%* This program is free software: you can redistribute it and/or modify
%* it under the terms of the GNU General Public License as published by
%* the Free Software Foundation, either version 2 of the License, or
%* (at your option) any later version.
%*
%* This program is distributed in the hope that it will be useful,
%* but WITHOUT ANY WARRANTY; without even the implied warranty of
%* MERCHANTABILITY or FITNESS FOR A PARTICULAR PURPOSE.  See the
%* GNU General Public License for more details.
%*
%* You should have received a copy of the GNU General Public License
%* along with this program.  If not, see <http://www.gnu.org/licenses/>.
%*
%*

\chapter[Strategy Tips]{{\Huge S}TRATEGY {\Huge T}IPS}

\index{tips}

\section{\textsf{Economic}}

\index{strategy!economic}

\textswab{\huge{T}}here are three things to remember if you want to build a healthy economy: trade, trade, and trade. Even a Kingdom with no access to natural resources will be able to prosper as long as it establishes trading links with as many other Kingdoms as possible.

A large population will give you a huge economic boost as it will provide a large tax base and will generate a large demand for goods. It will be difficult to get a large population, however, unless you supply a good standard of living for your Villagers. You do this by making sure that your Villagers have access to well-stocked Markets.

Special Taxes can also give your treasury a boost. You should set it up so that Special Taxes will be automatically collected in every one of your Villages when they reach a certain loyalty level.

If you take care of these basic things, you will soon have more than enough money for erecting Buildings, making Weapons, hiring Mercenaries, and paying Bribes.

\section{\textsf{Military}}

\index{strategy!military}

\textswab{\huge{T}}he training and fighting skills of your soldiers are greatly enhanced by having Generals with a high Leadership Level. Because of this, it is vital that you build up a good supply of skilled Generals.

Your enemies are of course thinking along the same lines, so it is to your advantage to relieve them of their Generals. When in battle, the first people that you should target are the enemy’s Generals. When they are dispatched, the fighting ability of the enemy force will be greatly diminished.

At the very beginning of the game, instead of just Recruiting Peasants for your first Troop, try Training eight of them in Leadership. This will give them double the strength and ability of raw recruits.

Fighting to the last man may be brave and honorable, but it is also stupid. It is far better to run away so that you may live to fight again, especially if you are about to lose some very well-trained soldiers.

Running away can sometimes be used as an effective ruse. Try running into the arms of a far larger force of your soldiers. If you are playing with the fog of war on, this method can deliver a nasty surprise to your enemy.

\section{\textsf{Espionage}}

\index{strategy!espionage}

\textswab{\huge{A}}t the very beginning of the game, send spies into as many Independent Villages as you can. There they will have time to increase their skill while sleeping. If, in the future, you are trying to absorb a Village, you can activate your Spy and use him to sow dissent. If another Kingdom takes over an Independent Village where your Spy is sleeping, he will be in good position to be drafted into their service.

If one of your Spies has been drafted into an opponent’s forces, use him to bribe his General. If you have the money, don’t worry about the cost, because with enemy Generals under your control you will have an important ace up your sleeve. If that Kingdom declares war on you, you may use your secretly controlled Generals to capture their Forts and quickly turn the tide of the war.

If one of the Generals that you have turned has a high leadership level, it may be a good decision to leave the Fort uncaptured in case the enemy’s King is slain in the war. In this situation, your Spy General may well become the next King and subsequently turn the enemy empire over to you.

If your Spy is alone in an enemy Fort with the enemy King or a high ranking General, it would be a good time to attempt an assassination. By being alone with the target, you will have a much better chance of succeeding.

When you are in an enemy building and selecting units to bribe, it is a good idea to first choose those with the lowest loyalty to their King.

If you have more than one Spy in the building, you will be asked which one will offer the bribe. Choosing the Spy with the highest skill will give you the greatest chance of success, but it will also open your most valuable Spy to the possibility of exposure and execution.

\begin{wrapfigure}{r}{0.1\textwidth}
    \vspace{-20pt}
    \begin{center}
        \includegraphics[width=0.1\textwidth]{Treveal}
    \end{center}
    \vspace{-20pt}
\end{wrapfigure}

If you wish one of your Spies to mimic defecting to the enemy without the enemy suspecting that he is a Spy, it may be wise to first change his cloak to Independent white. While in that cloak, place him near an Independent Village, \textbf{Click} on the \textbf{Surrender Tile}, and then change your Spy’s cloak to the enemy’s color.

When you follow this procedure, your enemy will be notified that an Independent Villager has joined his Kingdom. If you change the Spy’s color directly from yours, it is more likely that your enemy will suspect his motives and then execute him.

Use your Spies to sow dissent in enemy Villages. If you have one or several Spies in an enemy Village, watch for the loyalty level of that Village to fall to a low level. Then, all at once, change the task of your Spies in that Village from “Sleep” to “Sow Dissent”. Then sit back and watch the fun.

When a rebellion takes place, all of your Spies in the Village will exit and change their cloaks to the color of a Kingdom that is not at war with your enemy. When they are a safe distance from the fighting, they will become idle and wait for your further instructions.

\begin{wrapfigure}{r}{0.1\textwidth}
    \vspace{-20pt}
    \begin{center}
        \includegraphics[width=0.1\textwidth]{Tsneak}
    \end{center}
    \vspace{-20pt}
\end{wrapfigure}

When your Spy is toggled to “Sneak” and cloaked in the colors of your enemy, your units will not attack him because they will be aware of his true identity.

In the Sneak mode, you will also be able to control your Spy’s movements. You may in this way try to infiltrate your Spy into enemy Forts. This may be dangerous however because the computer or human opponents will be more likely to notice that this unit is moving without any order for him to do so. If they do notice, it will be the chop for him.

\section{\textsf{Small Kingdom Survival}}

\index{strategy!small kingdom survival}

\textswab{\huge{I}}f you are a small, weak Kingdom and wish to survive to become a big and powerful Kingdom, it is very important to make sure that you keep a good Reputation. This is because attacking a Kingdom with a high Reputation will severely lower the Reputation of the attacking Kingdom.

It is also a good idea to try to form an alliance with a Kingdom that has a high Reputation. This is because those Kingdoms will be less likely to break their alliance treaty with you.

Making trade treaties with as many Kingdoms as possible is also a good idea, both for your economy and for your survival. If you import a large amount of goods from a Kingdom or if you sell a large amount of goods to a Kingdom, they will be less likely to go to war with you.

Try to infiltrate as many Spies as you can into other Kingdoms, friend or foe. While you are still small and weak, your Spies will be increasing their skills and working their way into the society of the other Kingdoms.