%*
%* Seven Kingdoms: Ancient Adversaries
%*
%* Copyright 1997,1998 Enlight Software Ltd.
%* Copyright 2018 Timothy Rink
%*
%* This program is free software: you can redistribute it and/or modify
%* it under the terms of the GNU General Public License as published by
%* the Free Software Foundation, either version 2 of the License, or
%* (at your option) any later version.
%*
%* This program is distributed in the hope that it will be useful,
%* but WITHOUT ANY WARRANTY; without even the implied warranty of
%* MERCHANTABILITY or FITNESS FOR A PARTICULAR PURPOSE.  See the
%* GNU General Public License for more details.
%*
%* You should have received a copy of the GNU General Public License
%* along with this program.  If not, see <http://www.gnu.org/licenses/>.
%*
%*

\chapter{FAQ}

\section{FAQ} % Dummy section.

% CHANGED SECTIONS TO SUBSECTIONS TO FIX A HEADER ISSUE. BETTER TO CUSTOM WITH FANCYHDR OR ADD A SECTION MARKER

\index{frequently asked questions}

\subsection{How can I slow down or speed up the game?}

% kEYS HERE.

Press the 0-9 keys on your keyboard. 0 is pause, and 9 is the fastest. Many of the overlays and menus will still function with the game speed set to zero.

\subsection{Why is my Caravan Tile disabled? I am unable to hire any more Caravans!}

This could be because you do not have enough money or, more probably, because you do not have enough Villagers in your Empire to support any more Caravans. Each Caravan needs 10 Villagers to support it.

You will not lose your existing Caravans, however, if you lose some of the Villagers who are supporting them.

\subsection{Why can’t I drop off any goods in the Market of a Kingdom with which I have a trade treaty?}

If you were allowed to do this, the builders of a Market would quickly lose control over what goods are being placed there. Only the kingdom which owns a Market may sell its goods there.

% Note here.

If you have a Trade Treaty with a foreign Kingdom, you may instead build your own Market in their territory; a Market that is Linked to their Villages and into which you may drop off your goods for sale to the people of that Kingdom.

Note that the Villagers of foreign Kingdoms will purchase goods from their own Markets before they purchase goods from yours. So if you are planning to build a Market in a foreign Kingdom, make sure that you put on sale only those goods that the locals do not already have access to or of which they have an insufficient supply.

\subsection{Why is my Train Tile disabled? I am unable to train any more Peasants!}

This can occur if you have run out of money or, more likely, because there are no more Peasants remaining in your Village. Although there still may be many people living in your Village, they are all working at other jobs, and you will therefore be unable to train them.

\begin{wrapfigure}{r}{0.1\textwidth}
    \vspace{-20pt}
    \begin{center}
        \includegraphics[width=0.1\textwidth]{Tstar}
    \end{center}
    \vspace{-20pt}
\end{wrapfigure}

% Note the bold here.

Another reason is that you have no General or King in your Fort and, therefore, no way to issue orders to the Village. If your General has been slain in battle, you will need to promote a soldier to take his place. To do this, select a soldier with high leadership skills and \textbf{Click} on the \textbf{Star Tile}. He will then become a General.

\subsection{I can not find my King! How do I find him quickly?}

% F5 here. 'K' key here.

Press the F5 key or \textbf{Click} on the Military Scroll at the top of your screen. Then \textbf{Double-Click} on the King in the list of your military commanders. This will center your main screen over his location. For an even faster method, you can press the ‘\textbf{K}’ key to find and select your King.

\subsection{Why is my Mine producing nothing even though I have Miners inside?}

Either the Natural Resource has been depleted or the stock of mined Raw Materials has reached its maximum of 500. You must clear some of the stock before you can mine any more.

The same thing will happen if your Factory’s stock has reached 500. In this case, the Factory will stop producing Finished Goods.

\subsection{I have ordered a building to be built, but it wasn’t! Why not?}

This could happen for several reasons. First of all, it is possible that you have run out of money and cannot afford to build anything.

Secondly, the building placement square might have been black instead of white when you clicked on the location. If it is black, it means that the selected location is not open for building.

% might be changed.

Thirdly, when the builder got to the location, there may have been a mobile unit temporarily occupying the space.

\subsection{When I attack an Independent Village, I end up killing everybody! How can I do this and still leave some people alive?}

You should try being more subtle in your methods: You could try building a Fort, linked to the Village and staffed with a commander of the same nationality. This will help to lower the Village’s resistance before your attack, thus causing fewer casualties.

% Quotation marks here.

You could also try infiltrating a Spy or two of the same nationality as the Villagers. Set their task to “Sow Dissent” and then wait for the resistance to fall.

It is also important not to attack a Village with too few people.

\subsection{I have no more Peasants available to work in my new Mine, so I want to retire some from my Factory and then train them as Miners. But when I send them from the Factory back to the Village, they immediately go back to the Factory! How do I prevent this?}

Before you send them from the Factory back to the Village, you should close the Link between the two by \textbf{Clicking} on the rotating green and yellow circle. This will prevent them from volunteering to work in the Factory again.

\subsection{It seems that the positive and negative numbers showing my money statistics are not correct! My money total is increasing, but the incoming money is shown as a negative!}

This is because the income is shown for the preceding 365 days. It may be negative for that time period if you have paid for some expensive buildings or hired some costly Mercenaries.

\subsection{Why can’t I capture an enemy’s empty Fort simply by moving into it? This would seem the realistic thing to do!}

It might seem so, but when playing the game with others you would see that whenever you Sortied your troops for battle, another player would quickly send a unit in to capture your Fort. Now \textit{that} would not be very realistic.

Think of it this way: when your soldiers leave the Fort, they don’t leave it completely empty. There is still support staff there and the gates are securely locked. In this case, it would be far stranger if someone could just walk in and claim it as their own.

If you want to capture a Fort (or other structures), make use of your Spies! This way is much more fun!

\subsection{I tried to settle a person in my Village but he won’t go in! He just stands outside!}

All Villages have a population limit of 60. If your Village is full, you will be unable to settle anyone there.

% You will see... show?

In this case, it would be a good idea to settle a new Village next to the old; then \textbf{Click} on the old Village. You will see an icon of four pink arrows centered on the new Village. \textbf{Right-Clicking} on this will transfer Peasants from the old Village to the new, thus freeing up space for more population growth!

\subsubsection{Does it help to have more than one Tower of Science working on the same project?}

You bet! Two Towers with similarly skilled scientists will take half the time of a single Tower to finish a project. But why stop at two? Try three, four, or ten!

\subsection{When I send a soldier into an already full Fort, he replaces one inside who then stands outside. I understand this, but when I do it in a Factory or Mine, etc. nobody comes out! Does somebody just disappear?}

Nobody disappears. Unlike soldiers, who live in their Forts, workers in Factories and other buildings live in Villages. When you send someone into a full Factory, the lowest skilled worker in that Factory will resign and become a Peasant in his Village.

\subsection{I have Caravans carrying Raw Materials into my Factory but nothing is being produced! Yes, my Factory already has workers in it.}

\begin{wrapfigure}{r}{0.1\textwidth}
    \vspace{-20pt}
    \begin{center}
        \includegraphics[width=0.1\textwidth]{Tgoodcycling}
    \end{center}
    \vspace{-20pt}
\end{wrapfigure}

\textbf{Click} on your Factory and then Toggle the \textbf{Change Production Tile} until the Factory’s production is set to match the incoming Raw Material.

\subsection{When my troops hit an independent Village, the Villages sometimes come out to fight and sometimes not. How can I know which will happen?}

Whether they come out to fight or not will depend on their resistance level to your rule. Those Villagers who have a resistance level of more than 50 will come out to fight you.

You can check the \textit{average} resistance level of the Village by \textbf{Clicking} on it.

\subsection{There are Peasants still in my Village but I find myself unable to Recruit any of them!}

This will occur if their loyalty level is 30 or less. Wait until it goes up (or quickly raise it with a Grant) and then try again.

\subsection{Why can’t I fire on my Porcupines?}

% Bold key here.

You must hold down the \textbf{Shift key} when you target your Porcupine. This is because the \textbf{Right-Click}, by itself, is used to select Troops.

\subsection{Is there any risk to using people who have defected to me?}

Yes there is. It is always possible that a defector is in fact a Spy in the pay of an enemy. Generally speaking, units coming from Independent Villages are not Spies. It is possible, however, that a clever enemy may first disguise his Spy as an Independent before ordering him to defect to you.

\subsection{I see my Caravan standing still outside my Market. Why is it doing this?}

It may be because a Market at which you have set it to stop has either been destroyed or closed to you by your state of war with that Kingdom.

\subsection{How do I move my Village nearer to a Mine so that I can save on the Raw Material transportation time and cost?}

It is a good idea to build up Villages in the vicinity of Mines. The first thing that you should do is to build and staff a new Fort near to the Mine. You should then slowly Recruit as many Villagers as you can from your old Village and Settle them within Linking distance of the new Fort.

Because Recruiting causes a drop in the Villager’s Loyalty, it may be a good idea to Train some of the Villagers in Mining and Manufacturing instead. This is a good idea because you will need Miners and Manufacturers at the new Village anyway, and by Training them instead of Recruiting them, you will avoid the drop in Loyalty.

\subsection{Does it make a difference if your Peasants have a Loyalty Level of 40 as compared to 100? Do they produce more food or what?}

They do not produce more or work harder. The reason for wanting a high Loyalty Level is that your people will be less likely to revolt or defect, harder (and more expensive) to bribe, easier to tax, and generally easier to control.

\subsection{How can I quickly remove those news messages at the bottom of the screen?}

% Key here.

Press the X key on your keyboard.

\subsection{How can I quickly execute a person whom I suspect of being a Spy?}

% Bold here.

Select the unit and then \textbf{Click} on the \textbf{X icon} that you can see to the left of the unit’s hit-point bar. This is also useful getting rid of old caravans or unneeded immigrants.

\subsection{How can I train a Spy as a Soldier so that I may try to infiltrate him into an enemy Fort?}

You first train a Villager in Spying. When he exits the Village, send him into one of your Forts for a short time so that he can gain some soldiering skills. When you take him out of the Fort, he will be sporting two icons over his head: one showing that he is a Spy and one showing that he is also a Soldier.
